\section{Why do methodologies exist?}
\label{sec:meth_why}
To find literature related to why methodologies were created, a descriptive systematic literature review was conducted.
Following the steps for a systematic literature review allows this question to be answered in a reliable and repeatable manner. \cite{xiao_2017}

By following the steps from \cite{xiao_2017}, the focus of this review is to answer ``why methodologies exist''.
The established review protocol stated only articles, conference proceedings, book chapters will be considered.
The search criteria was for titles to include any of the following phrases:
\begin{itemize}
    \item development methodology
    \item software development process
    \item software development lifecycle
    \item software development life cycle
\end{itemize}

And to answer the question of ``why'', any of the following needed to appear in the abstract:
\begin{itemize}
    \item creation
    \item create
    \item purpose
    \item use
    \item benefit
\end{itemize}

With this review protocol, the following collections were searched:
\begin{itemize}
    \item ScienceDirect with 32 results
    \item ACM with 55 results
    \item Emarald insight with 13 results
    \item ProQuest with 126 results
    \item EBSCOHost with 102 results
    \item IEEE with 243 results
\end{itemize}

Springer was part of the search plan, but their search is broken when boolean operators are used in the title.
Therefore, Springer was cut from the searched publications.

The combined results contained 543 \pieter{Why doesn't this add up? 🤔} items from 1972 to 2022.
Two sets of screening was used to filter this list down.
First, the results were evaulated on their abstractions and duplicates were removed.
This created a short list of 50 results.
The second set of screening involved downloading the full article for this 50 results and evaluating their abstract, introduction, and conclusion for inclusion.
It was not possible to get the full article for 6 of these results, and 1 result only had its title and abstract in English.
The remaining 43 results were thus evaluated and given a rank from 1 to 5.
A rank of 5 was given to the articles most relevant for ``why methodologies exist'' and a rank of 1 for least relevant.

Since this is a descriptive systematic literature review, the qualities of each result was ignored to be able built an accurate historic picture of ``why methodoligies exist''.
However, it was decided to only analyse the results with a final ranking of 2 or more which was the case for 30 results.

Extracting the data involved making a list of phrases used to describe ``why a methodology exists'' or the benefits of a methodology from each result.
The results were then grouped into ten year periods with matching phrases being summed together.

\pieter{A backward search section is missing here}

The results will be discussed in chronological order, grouped in 10-year periods. \pieter{the first methodology was documented in 1956 but this search first result is from the 1970s}

\subsection{1970s}
In this period software development is still new with many practices being borrowed from engineering.
It was, therefore, found that the structured methodologies of the time promised to improve programmer's productivity.
For the rest of this review, we will refer to improving productivity as resource management.
Methodologies also tried to improve the reliability, maintainability, and quality of the end product \cite{yourdon_1977}.

In this period personal computers made their entrance for the first time.
However, engineers still wrote most software to perform a specific task, for example flying Project Apollo\pieter{what are the physical aspects of project apollo}.
These systems would have clearly defined procedures written by engineers that the software would need to follow.
Hence, why quality and reliability was important in this period.

At the same time it would not be possible to update the program flying Apollo in mid-flight. \pieter{qualify}
Therefore, maintainability refers more to the short-term maintainability that would be needed before take-off.

In this period, methodologies were not a complete lifecycle encompassing package as they are today \cite{soi_1982, beregi_1985}.
Rather, there was a separate methodology for design (like structured design) and another methodology for implementation (like structured programming) and yet another for analysis (structured analysis).
Using them together was optional, thus it was possible to swap structured programming for top-down development while still using structured design. \cite{yourdon_1977}

\subsection{1980s}
In the early 1980s it was proposed that a methodology would be more successful if it covered all the development phases \cite{soi_1982}.
And by the mid 1980s, these phase based methodologies and the tools they used would start to be combined.
The term methodology was retained to describe this combination of phases and tools \cite{beregi_1985}. \pieter{See if there is a thausuris tool}

When looking at all the phases of development, cost became a concern to manage and reduce \cite{vanderlei_1983, peacham_1985, loesh_1985}.
Managing time became important as it had an impact on cost and timeliness ensures the system is not obsolete by the time it is operational because it took too long to be developed \cite{peacham_1985, beregi_1985, mannino_1987, paul_1993}.
This completes the triad of cost, time, and quality being important, but for any project, only two could be chosen.\pieter{where has quality gone}\pieter{can only take away if there is something else about quality}

Futher to cost, time, and quality, methodologies were used to manage:
\begin{itemize}
    \item User needs \cite{peacham_1985}
    \item Maintainability \cite{peacham_1985}
    \item Identifying the correct problem to solve which we will call risk \cite{peacham_1985}
    \item Proper documentation of the system \cite{loesh_1985}
    \item Cooperation between developers, users, and management and to establish jargon for their communications \cite{loesh_1985}
    \item Overall resource management \cite{mannino_1987}
\end{itemize}

\subsection{1990s}
By the start of the early 1990s, methodologies which integrate the entire development process were trending.\pieter{check be verbs}
While this is happening, software projects become smaller with end user applications rising in popularity.
This causes the overall perspective to be more focused on collecting the correct requirements \cite{paul_1993} from the end users.
For example, end user satisfaction made an appearance for why a methodology is needed.
\cite{drake_1991}

Overall the reasons for using methodologies in the 1990s were to:\pieter{time, cost and quality}
\begin{itemize}
    \item Minimize risk \cite{drake_1991, trussel_1999}
    \item Reduce cost \cite{drake_1991, scarre_1992, paul_1993}
    \item Timely delivery \cite{drake_1991, scarre_1992, herald_1993, trussel_1999}
    \item Resource management \cite{drake_1991}
    \item Team management \cite{drake_1991}
    \item Scheduling \cite{drake_1991, paul_1993}
    \item End user satisfaction \cite{drake_1991}
    \item Senior management awareness \cite{drake_1991}
    \item Quality \cite{drake_1991, scarre_1992, herald_1993, trussel_1999}
\end{itemize}

The definition for quality is, however, starting to make a shift towards non-functional requirements and the quality of the methodology's documentation \cite{scarre_1992}.
And the need to establish criteria to evaluate the final software in earlier phases is identified \cite{paul_1993, herald_1993, grossman_1997}.

\unsure{I think my tense shifted from past tense. Which tense should I be writing in? When talking about the past? future imperfect}

\subsection{2000s and beyond}
For the last 20 years many methodologies were created for specific non-functional requirements such as performance and security.
Therefore, these results will not be considered as reasons why methodologies exist since correctly identifying and managing this single requirement is part of minimizing risk.

There were still 3 articles though that listed the following reasons for using a methodology: \pieter{same grouping as earlier} \pieter{a diagram/visual.... in summary}
\begin{itemize}
    \item Team management \cite{lucena_2008}
    \item Control scope \cite{lucena_2008}
    \item Timely delivery \cite{lucena_2008, garg_2015}
    \item Cost \cite{lucena_2008, uzturk_2013, garg_2015}
    \item Managing changing or critical requirements \cite{lucena_2008, uzturk_2013}
    \item Risk management \cite{lucena_2008, uzturk_2013, garg_2015}
    \item Planning, scheduling and velocity measurements \cite{uzturk_2013}
    \item Common communication language \cite{uzturk_2013}
    \item Makes each role aware of their responsibility and what to do next \cite{uzturk_2013}
    \item Quality \cite{garg_2015}
\end{itemize}

\subsection{When a methodology is not used}\unsure{I don't know if this will be unneeded fluff... discuss first}
