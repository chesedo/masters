\begin{abstract}
  At the end of the second world war we, for the first time, had access to a general computing machine in the form of turing machines.
  With them, we were able to create computer programs by manipulating stored binary digits.
  However, we would eventually need formal methods to plan and execute projects for creating computer programs.
  These methods became known as software development methodologies with the first one being peer reviewed in 1956 with many more to follow.
  Today dozens of software development methodologies have been proposed and this paper offers a framework to compare them to one another.
  This paper then presents a process for selecting the most appropriate methodology for a programming project based on the project constraints and desired outcomes.
\end{abstract}

\section{Introduction}
// Establish the reasons methodologies exists
\cite{einstein}

\begin{itemize}
    \item To deliver a project on date x
    \item To deliver quality software
    \item To lower costs
\end{itemize}

// Go on about how one can only really pick two of these.

// Therefore one needs to select the matching methodology after the two has been picked.

// Need KB, therefore need taxonomy, therefore need fca

// Since we are using FCA, will I need to cover its basics, or the basics for reading a laticce, here somewhere?

\section{Methodologies overview}
// Will need to use a "systematic literate review"

// Goes through the magor ones in chronological order

\begin{enumerate}
  \item Waterfall
  \item V model
  \item Shift left (possibly the same as the V model and DevOps)
  \item Agile
\end{enumerate}

\section{KB}
\section{Taxanomy}
\section{FCA}

\section{Building a knowledge base}
We build an initial context with basic attributes like, has fixed deadline, can adapt to changes, and testing first development.
Next, we add the set of most popular methodologies to this context.
Then we do an attribute exploration to confirm each methodology can be identified uniquely.
This is likely to add new attributes to our context.

Then we do an object exploration to add more methodologies to our context.
The attribute exploration followed by object exploration is done repeatedly until some x...

\section{Applying the knowledge base}

\section{Result}
\section{Conclusion}
No one perfect methodology.
We need to know the unique constraints of the project and then pick or mix up the appropriate methodology.
