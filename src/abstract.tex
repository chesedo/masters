\begin{abstract}
  At the end of the second world war we, for the first time, had access to a general computing machine in the form of turing machines.
  With them, we were able to create computer programs by manipulating stored binary digits.
  However, we would eventually need formal methods to plan and execute projects for creating computer programs.
  These methods became known as software development methodologies with the first one being peer reviewed in 1956 with many more to follow.
  Today dozens of software development methodologies have been proposed and this paper offers a framework to compare them to one another.
  This paper then presents a process for selecting the most appropriate methodology for a programming project based on the project constraints and desired outcomes.
\end{abstract}
